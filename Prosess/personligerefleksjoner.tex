\documentclass[../main.tex]{subfiles}

\chapter{prsonlige refleksjoner}
Kjetil
Kjetil har lært mer om hvordan drive med team arbeid med personer som har en annen studiebakgrunn, samt hvordan han kan bidra med sin kunskap til å drive prosjektet framover. Siden prosjektet var en helt ny ide, fikk han kjennskap til prosessen som inngår ved oppstart av en organisasjon, og hvordan gå frem når en ide skal realiserers. 

Utfordringer han har møtt har vært knyttet til finne løsninger på faglige problemer, så vel som takle stresse som kommer med et prosjekt av denne størrelsesordenen. En av de større utfordringen var at det tok tid før teamet ble helt enige om hva som skulle gjøres. Var til tider utfordrende å vite hva som var bestemt, siden gruppen hadde problemer med å ta avgjørelser.

Han har blitt bedre til å ikke spore av i samtaller.  Han har lært mer om å reflektere over situasjoner som oppstår og diskuterer det med de andre i gruppen. Han mener selv at han må bli bedre til å følge med på hvordan prosessen i en gruppe fungerer, så vel som å si i fra når ting ikke fungerer. 

Daniel
Hva jeg har lært
Han har lært seg mye innen ledelse, samt hvordan ha god  kommunikasjon i et team og sørge for at det eksisterer et tverrfagligsamarbeid. Han lærte at det var viktig å definere klare mål, som blir definer gjennom klar ledelse og effektiv organisering og kommunikasjon.

Han oppdaget at ledelse og kommunikasjon er tett relatert til hverandre, der mål måtte bli kommunisert tydelig til resten av teamet.

Mine utfordringer
Daniel fant det utfordrende å ta ledelsen, samt hvordan gi tilbakemeldinger. Han syntes at det var utfordrene at det mangelet struktur i arbeidet i begynnelsen av prosjektet. Det endret seg etterhvert som at gruppen ble mer kjent, og fikk mer erfaringen i teamarbeid. Siden Daniel ble valgt som leder ble det hans oppgave å delegere arbeidsoppgaver. Det ble en utfordring siden han var usikker på hva gruppen satt på av individuell kunnskap, samt hvordan kommunisere arbeidsoppgavene på en forstålig måte med klare mål.

Navn: Geir Kulia
Hvem er jeg?

Hva har jeg lært?
Forventningene Geir hadde til Eksperter i Team var høye. Ettersom han måtte skrive søknad om å bli med på gruppen forventer han også at det var en over gjennomsnittet engasjert gruppe. Det var aldri mangel på ambisjoner i gruppa, og i begynnelsen sto gruppen og stanget veldig med mange tanker, ideer og ting gruppen ville realisere, men ingen penger. Det hjalp derfor enormt etter at gruppen fikk sponsorer som var interesserte i å satse på oss. Dette gjorde også at vi fikk et ekstra ansvar. Gruppen var ikke lengre bare ansvarlige for oss selv og faget, men også ovenfor sponsorene.

Geir syntes at det var en utfordring å kombinere faget og organisasjonen. Spesielt det å ha rullerende roller var krevende, siden da viste han ikke hvem som hadde kontroll eller hvem han skulle henvise seg til. Kommunikasjonen internt i gruppa var mer utfordrende enn han hadde trodd på forhånd. 


Helge

Helge syntes at gruppen virket veldig positiv til eksperter i team, og var veldig motivert til å jobbe med prosjektet. Dette førte til at alle var positive da vi snakket om prosjektet og det dermed var veldig "Ja" stemning de første dagene. Det var ingen protest mot hvordan ting skulle gjøres og god stemning i teamet. Dette var bedre enn det han hadde forventet siden han hadde hørt fra studenter som hadde hatt faget før, at det ofte var uenigheter i starten.

Arbeidet gikk litt tregt i starten. Spesielt synes han at møtene var lite effektive i denne perioden. Helge lærte etter hvert at alle i gruppen måtte ta ansvar for å effektivisere møtene. I tillegg hjalp det med klarere rollefordeling. En annen ting han syntes at gruppen forbedret seg på var å velge hva som var egnet til møter og hva som kunne gjøres uten at vi hadde diskutert det i grupper på forhand. Han ble også mindre tålmodig til problemer i hvordan møtene ble holdt jo lengre inn i prosjektet vi kom. I starten godtok han at vi ikke var så veldig effektive, men etter ble han mer frustrert over problemene, selv om de var mye mindre enn det de var i starten. Han mener selv at han burde vært flinkere til å påpeke dette, noe han gjorde alt for sjelden, også i slutten av prosjektarbeidet.

Usikker på hva som skulle gjøres. Hadde selv veldig lite kunnskap om hva som skulle gjøres i starten av prosjektet. Håpet da at andre i gruppen visste mer om dette enn han selv og dermed ville ta initiativ og si hvordan man skulle gå fram. Dette lærte han etterhvert var feil innstilling og at han mener selv at han  burde ha tatt mer ansvar for å lære om dette selv siden de andre var omtrent like nye på dette området som han selv. 


Navn: Morten Liland

Litt om meg:

Hva har jeg fått ut av faget
Han har lært om vanskelighetene med å sammarbeide i tverrfaglige team, både med tanke på kommunikasjon, det å ta avgjørelser, og det å se helheten i et prosjekt før en avgjørelse tas. At det å sette seg konkrete mål er avgjørende både for fremgangen i prosjektet, og vanskelighetene ved å lede projektet  har kommet tydelig frem i løpet av EiT.

Utfordringer
Det har vært en utfordring å få på plass en struktur i arbeidet med prosess ved siden av prosjektet, men der har fasiliteringen vært til stor hjelp.