\documentclass[../main.tex]{subfiles}

\chapter{Helge}

Helge er 4.års materialteknologistudent med fordypning i materialutvikling og bruk og har mye kunnskap om alt som omandler både metaller, kompositter og polymerer. Han har tidligere ikke jobbet i en gruppe med et prosjekt som er av så omfattene karakter som EiT.

Helge forventet å lære like mye om seg selv som han kommer å lære om å jobbe i et team. Han forventer å få innsikt i hvordan han selv takler stress og ansvar, men også hvordan det er å jobbe med ting som han før EIT prosjektet hadde veldig lite kunnskap om. I tillegg forventer han å få bruke det han tidligere har lært i løpet av studiet til noe mer praktisk. Om gruppearbeid forventer han å lære om hvordan et effektivt team skal håndtere konflikter, hvordan det skal ta avgjørelser og generelt sett hvordan et effektivt team skal sammarbeide. Han hadde hørt rykter om at det det kunne bli store konflikter i en EIT Gruppe, men han håper at gruppen unngår dette.

Han valgte byggelandsbyen for å utfordre seg selv og lære seg om design-, byggin- og innkjøps-prosesser. Han mener dette er kunnskap en kun kan få gjennom et praktisk prosjekt som dette og samtidig var det sosialt og gøy.

Utfordringene han møtte var mest knyttet til hvordan effektiviteten i gruppa kan forbedres. Han lærte mer om hvordan han selv kan ta ansvar for at møter blir mer effektive og har blitt bedre til å oppdage problemer og forbedringsområder en gruppe har når det gjelder effektivitet. Selv om han mener at han burde være flinkere til å påpeke disse problemene over for de andre i gruppen.

Helge hadde fra før ikke noe erfaring i å være med i å danne en studentorganisasjon og lærte fort at det var mer arbeid enn han hadde forventet. Siden han ikke viste hvordan man går frem i opprettelsesfasen av en organisasjon, håpet han at de andre gruppemedemmene skulle ta ansvar for det som skulle gjøres. Dette har han lært var feil innstilling, siden de andre i gruppen også var nye på dette området. Han merket derfor at gruppen manglet overliggende plan. Etter å ha vært med i dette prosjektet har Helge fått en god del mer kunnskap om prosjektplanlegging i denne typen prosjekter og han føler at prosjektet ville være en god del enklere hvis gruppen skulle gjøre det på nytt.



