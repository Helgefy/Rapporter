\documentclass[../main.tex]{subfiles}

\chapter{Kjetil}
Kjetil er 23 år, og studere sitt fjerde år ved kybernetikk og robotikk med spesialisering innen navigasjon og fartøystyring.

Forventinger han hadde til faget var å se hvordan han selv fungerer i et prosjekt, og hvordan han bli bedre til å kommunisere bedre med andre som ikke har samme fag bakgrunn som han selv. Han hadde som resten av de andre i gruppen Byggelandsbyen som første valg for EIT, men i motsenting til de andre hadde han ikke søkt seg spesifikt inn på prosjektet.

I løpet av EIT har han lært å bli bedre til tverfaglig kommenikasjon, og utfordringer knyttet til å finne praktiske løsninger på et teoretisk problem. Han har fått større innsikt i hvordan han arbeider i en gruppe. Han har en tendens til å bli veldig opptatt med det han holder på med. Har også funnet ut at det å stille og bli stilt kritiske spørsmål om det som en holder på med kan bidra til finne problemer som gruppen ikke er klar over.

Utfordringer han har møtt har vært knyttet til finne løsninger på faglige problemer, så vel som takle stresse som kommer med et prosjekt av denne støresleordenen. En annen utfordring han har hatt er at han er den eneste i gruppen som vet noe om kontroll systemer, og hvordan et kontroll problem kan løses. Gruppa er dermed sterkt avhengig av løsningene hans, og mener at dette kan være en svakhet siden de andre gruppemedlemmene ikke kan motargumentere løsningene han kommer med. 

Han har blitt bedre til å ikke spore av i samtaller. Han har lært mer om programering av en mikrokontroller, samt hvordan løse problemer med begrenset kunnskap om problemet. Han har lært mer om å reflektere over situasjoner som oppstår og diskuterer det med de andre i gruppen.