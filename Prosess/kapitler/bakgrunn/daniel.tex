\documentclass[../main.tex]{subfiles}

\chapter{Daniel}

Daniel er 4.års materialteknologistudent med fordypning i materialutvikling og bruk. Han har også en bachelor i maskin som bakgrunn. Tidligere har han engasjert seg mye med studentprosjekter og frivillig arbeid, og synes dette var en spennende mulighet for å få erfaring innen organisajon og ledelse.

Allerede før EiT hadde han planer om å starte et studentprosjekt og hadde derfor store forventninger når det kom til EiT og kombinere det sammen med muligheten for å starte et prosjekt som kunne tilføre NTNU noe nytt og spennende.

Daniel har i løpet av EiT lært mye om vanskelighetene med å sammarbeide i tverrfaglige team, spesielt utfordringer knyttet til oppstarten av et prosjekt og unge organisasjoner. Han har fått større innsikt i hvordan han arbeider i en gruppe. Han har en tendens til å prøve og gjøre alt selv, men har utover prosjektet funnet ut at det å delegere oppgaver og spørre om hjelp når han sitter fast, kan redusere mye av stresset han føler. Samtidig som det gir andre gruppemedlemmer mer tilhørighet til gruppen når han delegerer de viktige oppgaver.

Utfordringer han har møtt har vært knyttet til å finne løsninger på faglige problemer, så vel som takle stresset og ansvaret som kommer med lederrollen i prosjekt av denne støresleordenen. En annen utfordring han har hatt er at han har en tendens til å være snill og forsiktig i måten han leder på, dette er gjenspeiler de innpakkede tilbakemeldingene som gruppa har fått. Han har senere innsett at dette ikke er bra og har derfor blitt mer oppmerksom på hvordan han gir klare og mindre innpakket tilbakemeldinger.

Han har blitt bedre til å ta ledelsen under møter og delegere oppgaver. Han har lært mer om å se helheten i et prosjekt, samt hvordan løse problemer med begrenset kunnskap om problemet. Han sitt viktigheten og nytten av å reflektere over situasjoner som oppstår og diskutere det med de andre i gruppen. Daniel påpeker at det har vært en stor utfordring for gruppa å bli enig om et konkret mål i starten og at det til tider har vert vanskelig å få alle med på samme steg i prosessen av arbeidet. Men her har landsbyleder og fasilitatorteamet vært til stor hjelp underveis. Daniel har sett på dette som en uvurderlig erfaring og selv om han har blitt mer komfortabel og bedre i lederrollen, kan han fortsatt bli bedre.