\documentclass[../main.tex]{subfiles}

\chapter{Morten}

Morten studerer elektronikk, med fordypning i nanoelektronikk og fotonikk, på fjerdeåret.
Han har engasjert seg i diverse frivillige studentorgansisajoner i løpet av studietiden sin på NTNU, men synes det er var veldig interessant og spennende å starte opp et
et så stort prosjekt som Vortex NTNU. \\

Allerede før Morten startet med faget EiT hadde han et blandet syn på faget, dette var på grunn av fortellinger fra andre studenter som hadde hatt faget tidligere.
Han valgte derimot å gå inn med en positiv holdning, og med forhåpninger om å lære noe nytt
og få utnyttet sine fagkunskaper i en praktisk sammenheng. \\

Morten mener han har lært om vanskelighetene med å sammarbeide i tverrfaglige team, både med tanke på komunikasjon, det å ta avgjørelser, og det å se helheten i et prosjekt før en avgjørelse tas. At det å sette seg konkrete mål er avgjørende både for fremgangen i prosjektet, og vanskelighetene ved å lede projektet
har kommet tydelig frem i løpet av EiT. \\

Morten påpeker at det har vært en stor utfordring å få på plass en struktur i arbeidet med prossess siden av faget, men at fasilitatorteamet har vært til stor hjelp.\\