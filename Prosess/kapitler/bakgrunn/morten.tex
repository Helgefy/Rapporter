\documentclass[../main.tex]{subfiles}

\chapter{Morten}

Navn: Morten Liland

Litt om meg:
Jeg studerer elektronikk, med fordypning i nanoelektronikk og fotonikk, på fjerdeåret.
Jeg har engasjert meg i diverse frivillige student organsisajoner. Det å starte opp et
et så stort prosjekt som dette virket for meg veldig interessant og spennende.

Forventninger til EiT og prosjektet
Før jeg startet med faget EiT hadde jeg et blandet syn på faget, grunnet fortellinger andre.
Jeg valgte derimot å gå inn med en positiv holdning, og med forhåpninger om å lære noe nytt
og få utnyttet mine fagkunskaper i en prektisk sammenheng.

Hva har jeg fått ut av faget
Jeg har lært om vanskelighetene med å sammarbeide i tverrfaglige team, både med tanke på komunikasjon,
det å ta avgjørelser, og det å se helheten i et prosjekt før en avgjørelse tas. At det å sette seg 
konkrete mål er avgjørende både for fremgangen i prosjektet, og vanskelighetene ved å lede projektet
har kommet tydelig frem i løpet av EiT.

Utfordringer
Det har vært en utfordring å få, på plass en stryktur i arbeidet med prossess siden av faget,
men har har fasiliteringen vært til stor hjelp.