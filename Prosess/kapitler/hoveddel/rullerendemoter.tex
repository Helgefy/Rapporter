\documentclass[../main.tex]{subfiles}

\chapter{Introduksjon}
Mangel på lederskap
Situasjon:
De første EiT dagene etter at samarbeidsavtalen var skrevet, byttet vi hver landsbydag på hvem som skulle ha ansvaret for å lede gruppa. Dette gjorde vi for å at skulle få prøve seg i ulike roller og utforske komfortsonen sin, som er et av læringsmålene til EiT. Vi hadde også gått gjennom en personlighetstest for å bevisstgjøre hvilke roller hver av gruppemedlemmene hadde, men resultatet fra denne testen tok vi med en håndfull neve salt.
 
Det tydelig frem at vi manglet en klar leder og ambisjonsmål da vi skulle prøve å finne en fellesløsning på design av roboten. Ettersom vi byttet leder for hver landsbydag endret også målene seg og det hadde ikke blitt presisert i samarbeidsavtalen hvilken funksjon eller krav roboten skulle oppfylle. For å få en effektiv gruppe må gruppen nå sine mål ved å formulere målene godt og ha mål som angår alle medlemmene (JJ kap 1). Gruppen hadde derfor vanskeligheter med å ta en rask og konkret beslutning.
 
Siden gruppen hadde lite erfaring med å bygge slike avanserte ting tidligere begynte alle bare å jobbe i hver sin ende uten å ha en klar framgangsmåte eller et felles mål å jobbe mot. Et eksempel på dette var da vi begynte å programmere før vi hadde blitt enige om hvilken plattform vi skulle bruke først. En effektiv problemløsningsprosess bør oppfylle to kriterier: Gruppen bruker en systematisk fremgangsmetode som passer med problemet den vil løse, og alle medlemmene fokuserer på samme steg i prosessen samtidig (Schwarz, 2002a). Her var gruppen altså ikke effektive og gruppen opplevde situasjonen som krevende og kaotisk. Noe som til tider førte til tydelig frustrasjon og lav gruppemoral.
 
Refleksjoner/teori
En grunn til at gruppen hadde vansker med å ta raske beslutninger kan ha vært fordi det var veldig tidlig i gruppearbeidet og det manglet en tydelig leder. Vi kjente ikke hverandre så godt enda, og var usikre og redde for å tråkke på hverandre i gruppen. Dette kjennetegner det som blir kalt for forming stage ifølge (Tuckman).
 
For effektive grupper er det viktig å ha klare mål og definerte roller (Schwarz-kap-2), og det var under en samtale med fasilitator som påpekte at vi manglet en leder i gruppen. Gruppen reflekterte videre på det som hadde blitt påpekt, og konkluderte med at lederrollen ikke var definert godt nok. Siden det var Daniel som hadde startet prosjektet og de andre måtte søke seg inn var det naturlig å velge Daniel som leder etter denne samtalen. Storming - Tuckman
 
Ettersom gruppen ble bedre kjent med hverandre og rollene til hvert medlem ble klarere Norming (Tuckman). Ble det enklere å ta en konkret beslutning, og fordele ansvarsområder i gruppa. Arbeidsflyten og samarbeidet i gruppen bedret seg betydelig etter hvert som gruppa ble mer komfortable i rollene som var satt og Daniel i lederrollen. For eksempel ble morgenmøtene kortere og mer effektive og gruppemoralen var mye høyere enn tidligere. Det var god motivasjonen for gruppen at det nå var klarere rammer satt for prosjektet og moralen i gruppa var mye høyere.
 
Selv om gruppen nå var mer klar over hva som skulle gjøres, var gruppemedlemmene til tider fortsatt avhengig av at leder viste vei og delegerte ut oppgaver som skulle gjøres Performing(Tuckman). Dette var noe gruppen kanskje slet mest gjennom hele prosjektet, og delegering av oppgaver og ansvarsområder fra leder kunne derfor vært bedre. Dette mener vi er viktig å ta med videre til senere gruppearbeid eller prosjekt. For selv om rollene og ansvarsområdene er satt, må en hele tiden passe på at alle får oppgaver som passer dem best. I følge Wheelan (2009) så vil gruppens prestasjoner lide når medlemmer blir tillagt roller eller oppgaver som ikke er passende.

Aksjoner
For å gjøre det enklere for leder å vite hva resten av gruppa syntes, hadde vi etter hver EiT dag positive og negative tilbakemeldinger på lederstil. Vi sørget videre for at rollen til leder hele tiden var klar og hva slags ansvarsområder han skulle ha. Vi fant ut at det var lurt at Lederen kunne delegere oppgaver videre ettersom det passet for å minske stresset og det var han som skulle lede morgenmøter og sende ut saksliste for å informere alle.
 
Vi ser at vi fortsatt har mye vi kan bli bedre på. Leder bør være strengere i å sette av tid til oppgaver og være mer bestemt i ledelsen for å få mer gjennomslag i beslutningene. Det må også tydeliggjøres for gruppa hva planen er å delegere oppgaver som er passende. Det kan være lurt å ha en dialog mellom leder og hver enkelt gruppemedlem å få frem klarheter. Leder kan være bedre på å gi ærlige og konstruktive tilbakemeldinger som ikke er pakket inn i søte ord.
 
Fra denne situasjonen har vi lært at det å ha flere ledere fungerer dårlig. Det skaper for mye kollektiv ansvar, noe som fører til mindre sannsynlighet for ansvarsfølelse for den enkelte gruppemedlem. For fremtidige og større prosjekt bør leder kun fokusere på å lede og ha overordnet ansvar for at ting blir gjort. Vi har lært at det å bli kjent med en ny gruppe mennesker hvor medlemmene var litt varsomme ovenfor hverandre, Tuckman (1965, i Johnson \& Johnson, 2006) kaller forming stage, som er en fase med usikkerhet i gruppen hvor medlemmene gjør seg kjent med reglene innad i gruppen og sin egen rolle. Vi fikk sett hvordan vi taklet tidspress i et prosjekt hvor det var veldig forskjellig sammensatt gruppe. Vi følte behovet for å ha en klar leder som kunne ta beslutninger når gruppen ikke var enige og ikke kom seg framover. Lederen er en nøkkelrolle i en effektiv gruppe og det er viktig at medlemmene er klar over hva den innebærer og at de støtter den (Schwarz, 2002a). Vi oppdaget også at selv om det er tidspress i et kompleks prosjekt er det alltid viktig at lederen hele tiden koordinerer og sørger for at alle medlemmene forstår situasjonen og oppgavene som skal løses. Det kan spare mye tid og frustrasjon underveis.
Det kan være nyttig å tenke litt hver for seg først, i stedet for en rotete idémyldring. Vi ser for oss at å ha litt tid med egne tanker kan hjelpe til å få bedre ideer, fordi man ikke blir forstyrret av de andre forslagene som kommer kontinuerlig. Dette var noe vi gjorde i de fleste beslutningsprosesser siden, noe vi følte bidro veldig positivt. Det vi lærte av denne situasjonen ønsket vi å ta med oss videre gjennom dette gruppearbeidet, spesielt behovet for en litt tydeligere lederrolle. Vi hadde utnevnt en leder, men ikke tydeliggjort ansvaret til denne rollen.  For å gjøre det lettere for lederen definerte vi dermed rollen tydeligere og ble enige i hva det innebar å ta ansvar for å lede gruppen videre, og ta beslutninger når vi ikke kom til enighet.

\subsection{en TING til}
Viktigheten av å ha en leder
Situasjon:
Tidlig inn i EiT opplegget gikk vi gjennom en personlighetstest for å bevisstgjøre hvilke roller hver av gruppemedlemmene hadde. Fordi vi ikke kjente hverandre så og ikke stolte helt på resultatene fra personlighetstesten, ble det bestemt å rullere på hvem som skulle ha lederrollen. Dette var for å finne den som passet best til å lede gruppa. 

Det var ulike oppfatninger om hvordan arbeidet burde gjøres og hva vi ønsket å oppnå, noe som førte til at gruppa ikke klarte å komme frem til noen konkrete beslutninger. Det kom mange gode forslag til tiltak under morgenmøtene, men det manglet alltid en beslutning. Dette resulterte i at vi bare begynte å jobbe i hver vår ende uten å ha en klar framgangsmåte eller et felles mål å jobbe mot. Vi opplevde situasjonen som kaotisk og det var til tider tydelig frustrasjon gruppen. 

Det var mye arbeid med ting som ikke var nødvendig fordi vi ikke klarte å identifisere problemet. 

Daniel tok på seg etter hvert leder rollen.

Vi klarte for eksempel ikke å komme til enighet om en helhetlig løsning av design på ROVen.



Teori
(Tuckmans fasemodell)
En effektiv problemløsningsprosess bør oppfylle to kriterier: Gruppen bruker en systematisk fremgangsmetode som passer med problemet den vil løse, og alle medlemmene fokuserer på samme steg i prosessen samtidig (Schwarz, 2002a).

Refleksjoner
Problemer: Dårlig ledelse, ansvarsfølelse, delegering av oppgaver, komme til helhetlige løsninger og overordnet situasjonsblikk, avsporinger. Ustrukturert, ikke klart hva som er planlagt. Uenigheter om tidsbruk og holde seg til avsatt tid på oppgaver. For mye kollektiv ansvar og det funker dårlig med mange ledere. 
For effektive grupper er det viktig å ha klare og definerte roller (Schwarz-kap-2). Dette var noe vi manglet i form av at lederrollen ikke var definert. 

Under en samtale med en av fasilitatorene kom det tydelig fram, at vår gruppe ikke hadde en samlet misjon, visjon eller et definert mål under samarbeidstalen. Noe som var grunnen til at gruppearbeidet ikke ble effektivt (Scwarz-kap-2). Hva som skulle gjøres med EiT og hva som skulle gjøres med organisasjonen ble ikke tydelig skilt fra hverandre.

Med rullerende lederskap virket det som om målet endret seg hele tiden fordi det var ikke samstemte og definerte mål. 

En må ta ansvar for å styre gruppen mot målet
Skapte mangel på rutine under morgenmøter.



Forbedringspotensialer: Leder kan være bedre på å gi tilbakemeldinger. Være mindre forsiktig og mer bestemt i ledelsen. Lederrollen bør kun fokusere på å lede og ha overordnet ansvar for at ting blir gjort.
Vi er ikke god på å sette tidsrammer.
snakker kanskje litt for snilt med hverander, må muligens være strengere i ledelsestil for å få mer gjennomslag i besluttningene.
Gruppen bør bli flinkere til å gi leder tilbakemeldinger.
Lederen må bli flinkere til å følge opp at ting blir gjort. 

Aksjoner (Tiltak)
Ved EiT dag 6 bestemte vi oss for å ha en fast leder, dette bedret arbeidsflyten og samarbeidet i gruppen betydelig. Lederen lager sesjonsplan og dirigerer møtet for å bedre flyt og effektivitet. Forslag om å sende ut møteagenda i god tid slik at alle kan komme forberedt til møtet og komme med evt. forslag til endringer i møteplan osv.
For å gjøre det lettere for lederen definerte vi ansvarsoppgaver som skulle utføres. Eksempel på dette var å planlegge møter, skrive saksliste, komme med tilbakemeldinger på oppgaver, stille kritiske spørsmål.
Gjorde det enklere for Daniel å formidle planer til resten av gruppa ved bruk av mail. Men alle må fortsatt bli flinkere til å sjekke mail for informasjon som blir sendt ut.
For å gjøre det lettere for lederen hadde vi etter hver EiT dag tilbakemelding på lederstilen. 
Hver enkelt skulle skrive ned sine mål og så diskutere det felles.

Problemer: Dårlig ledelse, ansvarsfølelse, delegering av oppgaver, komme til helhetlige løsninger og overordnet situasjonsblikk, mangel på fremdriftsplan. Sliter med avgjørelser i gruppen. Bedre på å gjøre beslutninger. Ikke konkret mål. Iverksette tiltak som har blitt foreslått.

Forbedringspotensialer: 


Evaluering
Gruppens arbeidsflyt endret seg betydelig etter at vi gikk over til å ha en leder. Vi synes at samarbeidet forbedret seg. 
Bedre til å planlegge og delegere oppgaver.



