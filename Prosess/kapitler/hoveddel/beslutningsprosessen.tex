\documentclass[../main.tex]{subfiles}

\chapter{Vurdering av beslutningsprosessen}

De første EiT dagene etter at samarbeidsavtalen var skrevet, byttet vi hver landsbydag på hvem som skulle ha ansvaret for å lede gruppa. Dette gjorde vi for å at skulle få prøve seg i ulike roller og utforske komfortsonen sin, som er et av læringsmålene til EiT. Vi hadde også gått gjennom en personlighetstest for å bevisstgjøre hvilke roller hver av gruppemedlemmene hadde, men resultatet fra denne testen tok vi med en håndfull neve salt. 
\\ \\ 
Siden ingen gruppen hadde erfaring med å bygge fjernstyrte undervannsroboter tidligere begynte alle bare å jobbe i hver sin ende uten å bli enige om en klar framgangsmåte eller et felles mål å jobbe mot. Et eksempel på dette var da Geir begynte å programmere før Morten og Kjetil hadde blitt enige om hvilken plattform de skulle bruke først. En effektiv problemløsningsprosess bør oppfylle to kriterier: Gruppen bruker en systematisk fremgangsmetode som passer med problemet den vil løse, og alle medlemmene fokuserer på samme steg i prosessen samtidig \cite{SchwarzA2002a}. Her var gruppen altså ikke effektive og gruppen opplevde situasjonen som krevende og kaotisk. Noe som til tider førte til tydelig frustrasjon og lav gruppemoral.
\\ \\ 
Det er viktig  for gruppen å nå sine mål ved å formulere mål og oppgaver som angår alle medlemmene. Ettersom vi byttet leder for hver landsbydag endret også målene seg. Dette kan ha noe med at vi hadde forskjellig oppfatning om hvordan vi ville gå frem. Gruppen vår hadde derfor vanskeligheter med å ta en rask og konkret beslutning. For å bedre på gruppens besluttsomhet bestemte vi oss for å revidere samarbeidsavtalen og velge en klar leder som kunne styre oss i rett retning. Vi valgte også å lage en liste over krav og funksjoner roboten skulle oppfylle.  
\\ \\ 
En annen grunn til at gruppen hadde vansker med å ta raske beslutninger kan ha vært fordi det var veldig tidlig i gruppearbeidet og det manglet en tydelig leder. Manglende kjennskap til hverandre resulterte i usikkerhet, samt redsel for å komme med kritikk. Dette kjennetegner det som blir kalt for forming stage ifølge (Tuckman).
 \\ \\ 
Gruppen reflekterte videre på det som hadde blitt påpekt, og konkluderte med at lederrollen ikke var definert godt nok. Siden det var Daniel som hadde startet prosjektet og de andre måtte søke seg inn var det naturlig å velge Daniel som leder etter denne samtalen. Storming - Tuckman
 \\ \\ 
Gruppens besluttsomhet og arbeidsflyt bedret seg utover prosjektet etter hvert som gruppen ble bedre kjent med hverandre,  så vel som rollene og ansvarsområdene til hvert medlem ble klarere Norming (Tuckman). For eksempel ble morgenmøtene kortere og mer effektive etter at Daniel tok på seg  lederrollen. Dette førte til at gruppemoralen ble mye høyere enn tidligere og det ble mye mer motiverende å arbeide.
 \\ \\ 
Selv om gruppen nå var mer klar over hva som skulle gjøres, var gruppemedlemmene til tider fortsatt avhengig av at leder viste vei og delegerte ut oppgaver som skulle gjøres Performing(Tuckman). Dette var noe gruppen kanskje slet mest gjennom hele prosjektet, og delegering av oppgaver og ansvarsområder fra leder kunne derfor vært bedre. Dette mener vi er viktig å ta med videre til senere gruppearbeid eller prosjekt. For selv om rollene og ansvarsområdene er satt, må en hele tiden passe på at alle får oppgaver som passer dem best. I følge Wheelan (2009) så vil gruppens prestasjoner lide når medlemmer blir tillagt roller eller oppgaver som ikke er passende.
\\ \\ 
For å gjøre det enklere for leder å vite hva resten av gruppa syntes, hadde vi etter hver EiT dag positive og negative tilbakemeldinger på lederstil. Vi sørget videre for at rollen til leder hele tiden var klar og hva slags ansvarsområder han skulle ha. Vi fant ut at det var lurt at lederen kunne delegere oppgaver videre ettersom det passet for å minske stresset og det var han som skulle lede morgenmøter, så vel som sende ut saksliste for å informere alle.
\\ \\  
Vi ser at vi fortsatt har mye vi kan bli bedre på. Lederen bør være strengere i å sette av tid til oppgaver og være mer bestemt i ledelsen for å få mer gjennomslag i beslutningene. Det må også tydeliggjøres for gruppa hva planen er å delegere oppgaver som er passende. Det kan være lurt å ha en dialog mellom leder og hver enkelt gruppemedlem å få frem uklarheter. Leder kan være bedre på å gi ærlige og konstruktive tilbakemeldinger som ikke er pakket inn i søte ord.
\\ \\  
Gjennom arbeidet med EIT har vi lært at det å ha flere ledere fungerer dårlig. Det skaper for mye kollektiv ansvar, noe som fører til mindre sannsynlighet for ansvarsfølelse for den enkelte gruppemedlem. For fremtidige og større prosjekt bør lederen kun fokusere på å lede og ha overordnet ansvar for at ting blir gjort. Vi har lært at det å bli kjent med en ny gruppe mennesker hvor medlemmene var litt varsomme ovenfor hverandre, Tuckman (1965, i Johnson & Johnson, 2006) kaller forming stage, som er en fase med usikkerhet i gruppen hvor medlemmene gjør seg kjent med reglene innad i gruppen og sin egen rolle. Vi fikk sett hvordan vi taklet tidspress i et prosjekt hvor det var veldig forskjellig sammensatt gruppe. Vi følte behovet for å ha en klar leder som kunne ta beslutninger når gruppen ikke var enige og ikke kom seg framover. Lederen er en nøkkelrolle i en effektiv gruppe og det er viktig at medlemmene er klar over hva den innebærer og at de støtter den (Schwarz, 2002a). Vi oppdaget også at selv om det er tidspress i et kompleks prosjekt er det alltid viktig at lederen hele tiden koordinerer og sørger for at alle medlemmene forstår situasjonen og oppgavene som skal løses. Det kan spare mye tid og frustrasjon underveis.
Det kan være nyttig å tenke litt hver for seg først, i stedet for en rotete idémyldring. Vi ser for oss at å ha litt tid med egne tanker kan hjelpe til å få bedre ideer, fordi man ikke blir forstyrret av de andre forslagene som kommer kontinuerlig. Dette var noe vi gjorde i de fleste beslutningsprosesser siden, noe vi følte bidro veldig positivt. Det vi lærte av denne situasjonen ønsket vi å ta med oss videre gjennom dette gruppearbeidet, spesielt behovet for en litt tydeligere lederrolle. Vi hadde utnevnt en leder, men ikke tydeliggjort ansvaret til denne rollen.  For å gjøre det lettere for lederen definerte vi dermed rollen tydeligere og ble enige i hva det innebar å ta ansvar for å lede gruppen videre, og ta beslutninger når vi ikke kom til enighet.





