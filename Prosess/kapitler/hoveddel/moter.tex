\documentclass[../main.tex]{subfiles}

\chapter{Møter og Rom}

Vi så fort at vi brukte mye tid på møter som ikke førte til en besluttning. De første to ladsbydagene gikk mye av tiden til møter. Vi opplevde at vi ikke greide å holde oss til et tema under diskusjoner og det var mye tid som gikk med på å diskutere saker som ikke hadde mye med EIT å gjøre.  I løpet av møtene ble det ikke tatt mange beslutninger. Derfor måtte hver diskusjon tas flere ganger.

Målet for møtegjennomføringen er at gruppen skal komme til besluttninger uten å bruke lang tid. Besluttningene skal flest mulige medlemmer være enig i. Grunnen til at ting ikke fungerte under møte kan ha mange årsaker. Gruppen prøvde å ta tiltak for å motvirke flere av problemene som gruppen så i vår gjenomføring av møter.

Et av problemene var at vi manglet struktur på møtene. Med dette menes at vi ikke hadde noen plan om hva som skulle diskuteres under møtet og at vi ikke hadde noen plan om hvordan vi skulle gå fram for å ta besluttningene sammtidig som vi ikke greide å fastsette hvor mye tid møtet ville ta. Det at vi manglet plan om hva som skulle diskuteres på møtet var det største av disse problemene. Det som skjedde var at vi hadde møte bare for å ha møte og vi brukte mye av møtetiden på å diskutere hva som skulle diskuteres på møtet. Et annet problem med dette var at når vi fant noe som skulle diskuteres, hadde ingen tenkt over problemet og derfor visste ingen om det var et stort problem og det var behov for å diskutere det. Dette medførte at vi bare diskuterte for fylle opp møtet i stedet for å gjøre noe nyttig.  Det var værst de første landsbydagene. Da ble stort sett hele dagen fulgt opp med møter og diskusjoner. For å få vekk dette problemet satte gruppen igang et tiltak der hvert møte skulle ha en saksliste som alle gruppemedlemmene skulle se på før møtet startet. Det er møtelederen sit annsvar å sende ut denne slik. Etter dette tiltaket ble antall temaer som ble diskutert mindre og møtene ble av den grunn korter og mer effektive. Grunnen til dette var blandt annet at det ble enklere å se om gruppen burde diskutere temaene før møtene startet slik at vi ungikk en situasjon der vi diskuterte et tema i lang tid og kom etter hvert til konklusjonen at dette ikke var et tema som var værdt å ta opp i plenum. Noe som deffinerte møtene de første dagene. Cohen(++ Meeting Design Characteristics and Attendee Perceptions of Staff/Team Meeting Quality http://psycnet.apa.org/journals/gdn/15/1/90.pdf) kommer fram til at det å publisere en møteplan før møtet gjennomføres og at alle deltakerne ser gjennom den, har positiv effekt på møtegjennomføringen. dette stemmer godt overens med gruppes erfaring.

Det at vi ikke visste hvordan vi skulle gå fram for å ta besluttninger. Gjorde at besluttninger ikke ble tatt og når besluttninger ble tatt ble de av og til ikke fulgt opp. I teamkontrakten ble det avtalt, i denne sammenhengen, at besluttninger skulle tas med flertallbestemmelse, men gruppen hadde ikke diskutert noe om hvor lenge en diskusjon skulle foregå før mann skulle ta en besluttning og om folk kunne skjære gjennom og si at vi hadde diskutert alle sidene ved problemet og det var på tide å ta en besluttning. En annen ting som kan ha ført til dette er at folk ikke følte annsvar for at møtet skulle føre frem og ansvar for at besluttningene skulle følges opp. Dette problemet mente gruppen ville løse seg selv ved at gruppen ble mer kjent med hverandre og samholdet økte og at medlemen i gruppen var obs på problemet. I følge Leana(A Partial Test of Janis' Groupthink Model: Effects of Group Cohesiveness and Leader Behavior on Defective Decision Making http://jom.sagepub.com/content/11/1/5.short) vil medlemene i en gruppe med godt samhold være mer direkte i hvordan de diskuterer temaer og ikke sensurere seg selv under diskusjonen. Dette vil medføre at gruppemedlemene ikke var redde for å komme med forslag til besluttninger og tiltak og vil være sikrere på sine egene forslag. En situasjon der forslag til tiltak blir diskutert mer fritt vil gi raskere og bedre besluttningsprosess, enn en situasjon der bare problemene blir diskutert, som var situasjonen i starten. Dessuten vil en gruppe der flest personer er involvert i besluttningsprosessen ifølge Schwarz(kompendium) være mer effektive. Da vil de fleste medlemen føle seg mer forpliktet til besluttningene og gruppen vil i følge teorien være mer effektiv.
 
Det å ha møteleder var et av punktene på teamkontrakten. Dette ble ikke fulgt opp første landsbydag etter teamkontrakten ble skrevet. Dette ble gruppen oppmerksom på da vi evaluerte dagen og vi bestemte oss for å skaffe en møteleder til neste møte. Det neste møtet ikke veldig mye bedre selv om gruppen hadde en møteleder. Problemet her kan ha vært at gruppen ikke hadde definert oppgaver til møtelederen og at medlemene i gruppen som ikke hadde fått rolllen som møteleder, forventet at møtelederen alene skulle sørge for at møtene blir bedre. Susan Wheelan(Kompendium) skriver at effektive team-medlemmer har i oppgave å ta ansvar for fasillitering i stedet for å skylde på lederen. Rollene møtelederen har består av å  lage liste over ting som skal diskuteres under møtet og å sørge for at alle får denne før møtet startet, sette av tidsbruk til møtet og å sørge for at dette blir fulgt opp og å styre ordet hvis det er behov for det. Etter at disse oppgavene ble deffinert ble ledelsen under møtet vesentlig bedre. 