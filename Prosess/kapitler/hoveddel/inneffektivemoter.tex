\documentclass[../main.tex]{subfiles}

\chapter{Inneffektive møter}
Vi så fort at vi brukte mye tid på møter som ikke førte til en beslutning. De første to landsbydagene gikk mye av tiden til møter. Vi opplevde at vi ikke greide å holde oss til et tema under diskusjoner og det var mye tid som gikk med på å diskutere saker som ikke hadde mye med EIT å gjøre.  I løpet av møtene ble det ikke tatt mange beslutninger. Derfor måtte hver diskusjon tas flere ganger.
\\ \\
Mangel på struktur preget møtene vår i begynnelen. Det eksisterte ikke en plan for hva som skulle diskuteres, hvor lang tid møte skulle ta eller hvordan beslutninger skulle tas.
Det at vi manglet plan om hva som skulle diskuteres på møtet var det største av disse problemene. Det som skjedde var at vi hadde møte bare for å ha møte og vi brukte mye av møtetiden på å diskutere hva som skulle diskuteres på møtet. Et annet problem med dette var at når vi fant noe som skulle diskuteres, hadde ingen tenkt over problemet og derfor visste ingen om det var et stort problem og om det var behov for å diskutere det. Dette medførte at vi bare diskuterte for fylle opp møtet i stedet for å gjøre noe nyttig.  Det var værst de første landsbydagene. Da ble stort sett hele dagen følgt opp med møter og diskusjoner.
\\ \\
For å få vekk dette problemet satte gruppen i gang et tiltak der hvert møte skulle ha en saksliste som alle gruppemedlemmene skulle se på før møtet startet. Det er møtelederen sitt ansvar å sende ut denne slik. Etter dette tiltaket ble antall temaer som ble diskutert mindre og møtene ble av den grunn korter og mer effektive. Grunnen til dette var blandt annet at det ble enklere å se om gruppen burde diskutere temaene før møtene startet slik at vi unngikk en situasjon der vi diskuterte et tema i lang tid og kom etter hvert til konklusjonen at dette ikke var et tema som var verdt å ta opp i plenum. Noe som definerte møtene de første dagene. Cohen(++ Meeting Design Characteristics and Attendee Perceptions of Staff/Team Meeting Quality http://psycnet.apa.org/journals/gdn/15/1/90.pdf) kommer fram til at det å publisere en møteplan før møtet gjennomføres og at alle deltakerne ser gjennom den, har positiv effekt på møtegjennomføringen. Dette stemmer godt overens med gruppes erfaring. 
\\ \\
Vi hadde problemer med å komme frem til besluttninger. Det kan komme fra at vi ikke visste hvordan vi skulle gå frem for å ta beslutninger. Dette gjorde at beslutningene ikke ble tatt og når beslutninger ble tatt ble de nødvendigvis ikke fulgt opp.
\\ \\
I samarbeidsavtalen ble det avtalt at beslutninger skulle tas med flertallbestemmelse, men gruppen hadde ikke fastsatt en øvre grense på hvor lenge en diskusjon kunne vare. Samarbeidsavtalen mangle elementer som tilatter at folk kunne skjære igjennom en diskusjon for å si at nå må en beslutning tas. En annen ting som kan ha ført til dette er at folk ikke følte ansvar for at møtet skulle føre frem og ansvar for at beslutningene skulle følges opp. Dette problemet mente gruppen ville løse seg selv ved at gruppen ble mer kjent med hverandre og samholdet økte og at medlemen i gruppen var obs på problemet. I følge Leana(A Partial Test of Janis' Groupthink Model: Effects of Group Cohesiveness and Leader Behavior on Defective Decision Making http://jom.sagepub.com/content/11/1/5.short) vil medlemene i en gruppe med godt samhold være mer direkte i hvordan de diskuterer temaer og ikke sensurere seg selv under diskusjonen. Dette vil medføre at gruppemedlemmene ikke var redde for å komme med forslag til beslutninger og tiltak og vil være sikrere på sine egne forslag. En situasjon der forslag til tiltak blir diskutert mer fritt vil gi raskere og bedre beslutningsprosess, enn en situasjon der bare problemene blir diskutert, som var situasjonen i starten. Dessuten vil en gruppe der flest personer er involvert i beslutningsprosessen ifølge Schwarz(kompendium) være mer effektive. Da vil de fleste medlemen føle seg mer forpliktet til beslutningene og gruppen vil i følge teorien være mer effektiv.
 \\ \\
Det å ha møteleder var et av punktene på teamkontrakten. Dette ble ikke fulgt opp første landsbydag etter teamkontrakten ble skrevet. Dette ble gruppen oppmerksom på da vi evaluerte dagen og vi bestemte oss for å skaffe en møteleder til neste møte. Det neste møtet ble ikke veldig mye bedre selv om gruppen hadde en møteleder. Problemet her kan ha vært at gruppen ikke hadde definert oppgaver til møtelederen og at medlemene i gruppen som ikke hadde fått rolllen som møteleder, forventet at møtelederen alene skulle sørge for at møtene blir bedre. Susan Wheelan(Kompendium) skriver at effektive team-medlemmer har i oppgave å ta ansvar for fasilitering i stedet for å skylde på lederen. Rollen møtelederen har består av å  lage liste over ting som skal diskuteres under møtet og å sørge for at alle får denne før møtet startet, sette av tidsbruk til møtet og å sørge for at dette blir følgt opp og å styre ordet hvis det er behov for det. 
Etter at disse oppgavene ble definert ble ledelsen under møtet vesentlig bedre. 
\\ \\
Vi hadde problemer med å finne et rom der vi kunne jobbe på landsbydagene. Dette kommer av at det er mange EIT-grupper som har lyst på rom på onsdager. Vi greide av og til å bestille rom for en halv dag. For så å måtte bytte rom senere. Da vi ikke fikk rom bestemte vi oss for å jobbe i leiligheten til to av medlemmene.I følge Cohen(++ Meeting Design Characteristics and Attendee Perceptions of Staff/Team Meeting Quality http://psycnet.apa.org/journals/gdn/15/1/90.pdf) vil et bra møtelokale være et med få distraksjoner. Dermed vil det å jobbe hjemme hos noen gir dårligere effektivitet enn å jobbe i et rom på NTNU. men siden vi ikke fant et bedre arbeidsted, var dette den beste løsningen.