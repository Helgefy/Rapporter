\documentclass[../main.tex]{subfiles}

\chapter{Introduksjon}
Kjetil er 23 år, og studere sitt fjerde år ved kybernetikk og robotikk med spesialisering innen navigasjon og fartøystyring.
Forventinger han hadde til faget var å finne ut hvordan han selv fungerer i et team, og hvordan han bli bedre til å kommunisere bedre med andre som ikke har samme fag bakgrunn. Han hadde som de resten av gruppen Byggelandsbyen som første valg for EIT.

Daniel studerer 4.års Materialteknologistudent med bachelor i maskin, med fordypningen innen materialutvikling og bruk. Han har engasjert seg mye med studentprosjekter tidligere, og synes dette var en spennende mulighet til å få erfaring innen organisasjon og ledelse.
Han hadde store forventninger til EiT, og så på det som en mulighet for å starte et prosjekt som kunne tilføre NTNU noe nytt og spennende.

Geir går i fjerde klasse som sivilingeniør i elektronikk. Fordypningen hans er design av digitale systemer. Tidligere har han engasjert seg i studentprosjekter og organisasjoner, derfor synes ham at det å lage en ny organisasjon er noe han synes var veldig spennende og lærerikt. 

Helge Fylkesnes er 22 år og studerer materialteknologi, med fordypning innen materialutvikling og bruk. Fordypningen gir kunskaper om mekaniske og elektorkjemiske egenskaper til materialer. Han har hatt fag som omhandler både metaller, kompositter og polymerer.
Han har ikke tidligere jobbet i en gruppe med et prosjekt som er så langtgående og omfattene som prosjektet gruppen skulle jobbe med i EiT. Derfor forventet han å lære en del om teamarbeid i løpet av semesteret. Han forventet å lære like mye om hvordan seg selv, og hvordan jobbe i team. Om seg selv forventer han å lære om blant annet hvordan han takler stress og ansvar, men også om hvordan det er å jobbe med ting han i utgangspunktet hadde lite kunnskaper om. Han forventet å få brukt det han hadde lært i løpet av studiet på en praktisk måte. Om gruppearbeid forventet han lære om hvordan gruppen skulle håndtere konflikter, samt hvordan gruppen skulle ta avgjørelser og generelt sett hvordan et effektivt team skal samarbeide. Han hadde hørt at det kunne bli større konflikter på gruppene i faget, men håpet selv at gruppen kunne unngå større konflikter. Siden formålet med prosjektet vårt var å bygge en ROV, forventet han å lære mye om design-, bygging- og innkjøps - prosesser. Dette er kunnskap han tror mann ikke kan få uten å jobbe med et slikt prosjekt. Hans største håp var at han kunne utfordre seg selv og å ha det gøy med dette prosjektet.

Morten studerer elektronikk, med fordypning i nanoelektronikk og fotonikk, på fjerdeåret.
Han har engasjert seg i diverse frivillige student organsisajoner. Det å starte opp et
et så stort prosjekt som dette virket for han veldig interessant og spennende.
Før han startet med faget EiT hadde han et blandet syn på faget, grunnet fortellinger andre.
Han valgte derimot å gå inn med en positiv holdning, og med forhåpninger om å lære noe nytt
og få utnyttet sine fagkunskaper i en praktisk sammenheng.

I forbindelse med EIT har gruppen bestemt seg for å etablere en organisasjon samt bygge en ROV. Prossensraporten vil ta for seg hvordan prosesen har vært, og hvordan vi håndterte situasjoner med tiltak. Tiltakene vil bli sett i lys av teorien som finnes innenfor de situasjonene som tiltakene er ment mot.